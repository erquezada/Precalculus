\documentclass{article}
\usepackage{amsmath} % for mathematical symbols and equations
\usepackage{amsfonts} % for mathematical fonts
\usepackage{amssymb} % for mathematical symbols
\usepackage{tikz} % for drawing figures
\usepackage{graphicx} % for including images

\title{Precalculus - Chapter 1: Introduction to Functions}
\author{Eric Quezada}
\date{\today}

\begin{document}

\maketitle

\section{Rectangular Coordinates}

\subsection{Introduction}

Rectangular coordinates, also known as Cartesian coordinates, provide a way to represent points in a two-dimensional plane using ordered pairs $(x, y)$. This system is essential for visualizing functions and equations in precalculus.

\section{The Cartesian Plane and Coordinate Points}

\subsection{Introduction}

The \emph{Cartesian plane} is a fundamental concept in mathematics that provides a framework for representing and analyzing functions. It consists of two perpendicular axes, the \emph{x-axis} and the \emph{y-axis}, which intersect at the \emph{origin} $(0, 0)$.

\subsection{Coordinate Points}

A \emph{coordinate point} in the Cartesian plane is represented as $(x, y)$, where $x$ is the horizontal distance from the origin (along the x-axis) and $y$ is the vertical distance (along the y-axis).

\subsection{The Four Quadrants}

The plane is divided into four quadrants:

\begin{enumerate}
    \item \textbf{Quadrant I}: Located in the upper right, where both $x$ and $y$ are positive.
    \item \textbf{Quadrant II}: Located in the upper left, where $x$ is negative, and $y$ is positive.
    \item \textbf{Quadrant III}: Located in the lower left, where both $x$ and $y$ are negative.
    \item \textbf{Quadrant IV}: Located in the lower right, where $x$ is positive, and $y$ is negative.
\end{enumerate}

\subsection{Pythagorean Theorem}

The \emph{Pythagorean theorem} is a fundamental result that relates the lengths of the sides of a right triangle:

\[
c^2 = a^2 + b^2
\]

where:
\begin{align*}
    c & : \text{the length of the hypotenuse} \\
    a & : \text{the length of one of the other sides} \\
    b & : \text{the length of the remaining side}
\end{align*}

The Pythagorean theorem is essential for calculating distances between points in the Cartesian plane.

\subsection{Translating Points}

Translating points in the Cartesian plane involves moving them horizontally and vertically. When you translate a point by a certain amount in the $x$-direction and a certain amount in the $y$-direction, you change its coordinates accordingly. For example, translating a point $(x, y)$ by $(a, b)$ results in the new point $(x + a, y + b)$.

\section{The Equation of a Circle}

\subsection{Introduction}

In the Cartesian plane, a circle is a set of points that are equidistant from a central point called the \emph{center}. The equation of a circle with center $(h, k)$ and radius $r$ is given by:

\[
(x - h)^2 + (y - k)^2 = r^2
\]

\subsection{Significance}

The equation of a circle is significant in geometry and mathematics because it represents a fundamental geometric shape. It allows us to describe and analyze circles and their properties, such as finding the center, radius, and determining whether a point is inside or outside the circle.

The equation also has connections to the Pythagorean theorem. If you consider a point $(x, y)$ on the circle, the expression $(x - h)^2 + (y - k)^2$ calculates the square of the distance between the point and the center $(h, k)$. When this squared distance is equal to the square of the radius $r^2$, it confirms that the point lies on the circle.


\section{Graphs of Equations}

Equations can be graphed in the Cartesian plane, allowing us to visualize and analyze their solutions and relationships. This section explores the graphical representation of equations.

\section{Linear Equations in Two Variables}

Linear equations in two variables have the form $ax + by = c$, where $a$, $b$, and $c$ are constants. Understanding these equations is fundamental as they often represent real-world relationships.

\section{Functions}

A \emph{function} is a rule or relationship that assigns each input value to exactly one output value. We often represent functions using the notation $f(x)$, where $x$ is the input variable, and $f(x)$ is the output.

\subsection{Function Notation}

\begin{itemize}
    \item $f(x)$: The value of the function $f$ at $x$.
    \item $x$ is the \emph{independent variable}.
    \item $f(x)$ is the \emph{dependent variable} or \emph{output}.
\end{itemize}

\section{Analyzing Graphs of Functions}

Analyzing the graphs of functions provides insights into their behavior, including domain, range, and important features like intercepts and symmetry.

\section{A Library of Parent Functions}

Various parent functions serve as fundamental building blocks for more complex functions. These include linear, quadratic, exponential, and trigonometric functions.

\section{Transformation of Functions}

Functions can be transformed by shifting, stretching, or reflecting them. Understanding these transformations is crucial for analyzing more complex functions.

\section{Inverse Functions}

Inverse functions reverse the action of a given function. This section explores the concept of inverse functions and their properties.

\section{Mathematical Modeling and Variation}

Mathematical modeling involves using functions to describe real-world phenomena. We'll discuss how functions are used to model and understand various situations.

\section{Combinations of Functions: Composite Functions}

Composite functions are formed by combining two or more functions. We'll explore how to define, evaluate, and analyze composite functions.

\section{Conclusion}


In this chapter, we have covered a wide range of topics, including rectangular coordinates, graphs of equations, linear equations in two variables, functions, analyzing graphs of functions, a library of parent functions, transformation of functions, inverse functions, mathematical modeling and variation, and combinations of functions (composite functions). We also explored the Cartesian plane, coordinate points, the Pythagorean theorem, the translation of points, and the equation of a circle. These concepts are fundamental in precalculus and provide a solid foundation for more advanced topics in mathematics.

\newpage % Start a new page for homework problems

\subsection{Chapter 1 Homework Problems}

\subsubsection{Problem 1}

Solve the following equation for $x$:
\[2x + 5 = 11\]

\subsubsection{Problem 2}

Find the domain of the function $f(x) = \sqrt{4x - 1}$.

\subsubsection{Problem 3}

Graph the linear equation $y = -3x + 2$ in the Cartesian plane.

\subsubsection{Problem 4}

Determine whether the following function is even, odd, or neither:
\[f(x) = x^3 - 2x\]

\subsubsection{Problem 5}

Use the Pythagorean theorem to find the distance between the points $A(3, 4)$ and $B(7, 1)$ in the Cartesian plane.
\end{document}
